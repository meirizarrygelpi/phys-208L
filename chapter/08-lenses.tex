% Copyright 2018-2019 Melvin Eloy Irizarry-Gelpí
\setcounter{chapter}{7}
\chapter{Thin Lenses}
%%%%%%%%%%%%%%%%%%%%%%%%%%%%%%%%%%%%%%%%%%%%%%%%%%%%%%%%%%%%%%%%%%%%%%%%%%%%%%%%
In this experiment you will learn about the properties of images produced by thin lenses.
%%%%%%%%%%%%%%%%%%%%%%%%%%%%%%%%%%%%%%%%%%%%%%%%%%%%%%%%%%%%%%%%%%%%%%%%%%%%%%%%
\section{Preliminary}
%%%%%%%%%%%%%%%%%%%%%%%%%%%%%%%%%%%%%%%%%%%%%%%%%%%%%%%%%%%%%%%%%%%%%%%%%%%%%%%%
Mirrors can be used to redirect the path of light rays. When a ray encounters a mirror, it bounces off. Lenses can also be used to change the direction of light rays. However, a lens uses a different mechanism called \textbf{refraction}.

Refraction is what happens when a ray of light crosses the boundary between two different materials. For example, a light ray traveling through air and encountering a piece of glass. As the ray crosses from air to glass, its direction of travel changes.

Just like for mirrors, the size and shape of a lense affects how it refracts rays of light. You can have concave, convex, or planar (flat) lenses. In the previous experiment, you produced a real image on a screen with a concave mirror. Here you are going to use convex lenses to also produce real images. As for mirrors, you have the distance between the object and the lens ($d_{O}$), the distance between the image and the lens ($d_{I}$), and the focal length of the lens ($f$). These three quantities are related via
\begin{equation}
    \frac{1}{d_{O}} + \frac{1}{d_{I}} = \frac{1}{f}
\end{equation}
This is the same relation that you found to hold for the concave mirror. The image might have a different size and orientation as the object, so it is useful again to work with the magnification quantity:
\begin{equation}
    m = \frac{h_{I}}{h_{O}} = \frac{d_{I}}{d_{O}}
\end{equation}
Here, as before, $h_{I}$ is the height of the image, and $h_{O}$ is the height of the object.
%%%%%%%%%%%%%%%%%%%%%%%%%%%%%%%%%%%%%%%%%%%%%%%%%%%%%%%%%%%%%%%%%%%%%%%%%%%%%%%%
\section{Experiment}
%%%%%%%%%%%%%%%%%%%%%%%%%%%%%%%%%%%%%%%%%%%%%%%%%%%%%%%%%%%%%%%%%%%%%%%%%%%%%%%%
The experiment with the convex lens is very similar to the experiment done with the concave mirror. The only difference is that, since the image from the convex lens is behind the lens, you can use a full screen to project the image instead of the half-screen used before with the mirror. You have two convex lenses available, so you can measure the focal length of both.
%%%%%%%%%%%%%%%%%%%%%%%%%%%%%%%%%%%%%%%%%%%%%%%%%%%%%%%%%%%%%%%%%%%%%%%%%%%%%%%%
\section{Analysis}
%%%%%%%%%%%%%%%%%%%%%%%%%%%%%%%%%%%%%%%%%%%%%%%%%%%%%%%%%%%%%%%%%%%%%%%%%%%%%%%%
Although the experiment is slightly different, the analysis of the data is exactly the same as done for the concave mirror data. With the position of the object ($x_{O}$), the position of the lens ($x_{L}$), and the position of the image ($x_{I}$), you can calculate the two relevant distances:
\begin{align}
    d_{O} = \left\vert x_{O} - x_{L} \right\vert && d_{I} = \left\vert x_{I} - x_{L} \right\vert
\end{align}
The chart with $1/d_{I}$ in the vertical axis, and $1/d_{O}$ in the horizontal axis should have a linear shape. The slope should be:
\begin{equation}
    \text{slope} = -1
\end{equation}
The intercept should correspond to the inverse of the focal length of the lens:
\begin{equation}
    \text{intercept} = \frac{1}{f}
\end{equation}
Taking the inverse gives the focal length:
\begin{equation}
    f = \frac{1}{\text{intercept}}
\end{equation}
You can do this for both lenses. You can also calculate the magnification in two different ways and show that both ways are essentially equivalent.
%%%%%%%%%%%%%%%%%%%%%%%%%%%%%%%%%%%%%%%%%%%%%%%%%%%%%%%%%%%%%%%%%%%%%%%%%%%%%%%%
\section{My Data}
%%%%%%%%%%%%%%%%%%%%%%%%%%%%%%%%%%%%%%%%%%%%%%%%%%%%%%%%%%%%%%%%%%%%%%%%%%%%%%%%
...
%%%%%%%%%%%%%%%%%%%%%%%%%%%%%%%%%%%%%%%%%%%%%%%%%%%%%%%%%%%%%%%%%%%%%%%%%%%%%%%%
\section{Your Data}
%%%%%%%%%%%%%%%%%%%%%%%%%%%%%%%%%%%%%%%%%%%%%%%%%%%%%%%%%%%%%%%%%%%%%%%%%%%%%%%%
You should have positions and heights for both lenses.
%%%%%%%%%%%%%%%%%%%%%%%%%%%%%%%%%%%%%%%%%%%%%%%%%%%%%%%%%%%%%%%%%%%%%%%%%%%%%%%%
\newpage
\section{Your Lab Report}
%%%%%%%%%%%%%%%%%%%%%%%%%%%%%%%%%%%%%%%%%%%%%%%%%%%%%%%%%%%%%%%%%%%%%%%%%%%%%%%%
...
%%%%%%%%%%%%%%%%%%%%%%%%%%%%%%%%%%%%%%%%%%%%%%%%%%%%%%%%%%%%%%%%%%%%%%%%%%%%%%%%
\newpage
\section{Tables}
%%%%%%%%%%%%%%%%%%%%%%%%%%%%%%%%%%%%%%%%%%%%%%%%%%%%%%%%%%%%%%%%%%%%%%%%%%%%%%%%
...
%%%%%%%%%%%%%%%%%%%%%%%%%%%%%%%%%%%%%%%%%%%%%%%%%%%%%%%%%%%%%%%%%%%%%%%%%%%%%%%%
\newpage
\section{Figures}
%%%%%%%%%%%%%%%%%%%%%%%%%%%%%%%%%%%%%%%%%%%%%%%%%%%%%%%%%%%%%%%%%%%%%%%%%%%%%%%%
...